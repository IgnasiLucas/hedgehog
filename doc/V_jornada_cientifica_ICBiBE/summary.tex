\documentclass{article}
\usepackage[T1]{fontenc}
\usepackage[utf8]{inputenc}
\author{Kristýna Eliá\v{s}ová$^1$, J. Ignacio Lucas Lledó$^2$,\\Pavel Hulva$^1$, Barbora \v{C}erná Bolfíková$^3$}
\title{Secondary contact of hegdehogs in central Europe}
\date{\vspace{-5ex}}
\begin{document}
\maketitle
\noindent 1. Charles University in Prague, Prague, Czech Republic.\\
\noindent 2. Population Genomics (E3). Instiut Cavanilles de Biodiversitat i Biologia Evolutiva.\\
\noindent 3. Czech University of Life Sciences Prague, Prague, Czech Republic.\\

Secondary contact of previously diverged lineages gives a chance for admixture,
with different potential evolutionary outcomes. European hedgehogs (\emph{Erinaceus europaeus})
and Northern white-breasted hedgehogs (\emph{E. roumanicus}) are a classical example of postglacial
recolonization with an extensive contact zone in central Europe. In this study we use genome-wide
markers (SNPs via RADseq) and a pan-European sampling design to examine the detailed population structure
of these species. We reveal a deeper genetic structure in \emph{E. europaeus}, related to historical isolation
in different refugia. We confirm a low rate of hybridization between the two species, and we are able
to compare the signal of historical introgression with the actual genomic blocks of introgression
in one hybrid individual. Our detailed analysis help explain the evolutionary forces that shape
the population structure and the interactions between European and Northern white-breasted hedgehogs.

\vspace*{1cm}
Keywords: \emph{Erinaceus europaeus}, \emph{Erinaceus roumanicus}, \emph{Erinaceus concolor},
contact zone, hybridization, introgression, genetic structure, RADseq, SNPs, postglacial recolonization.

\end{document}
